\documentclass{beamer}

\usepackage[utf8]{inputenc}
\usepackage{default}

\begin{document}
\begin{frame}
\frametitle{DAT171 - Objektorienterad programmeringsteknik i Python}
\begin{itemize}
 \item Ny kurs! Java $\to$ Python
 \item 3 inlämningsuppgifter     (3.0hp)
 \item Tentamen i datorsal       (4.5hp)
 \item Förkunskapskrav: Grundläggande kunskap i programmering, såsom variabler, villkor, loopar, och funktioner i något programmeringsspråk.
 \item Kurslitteratur: \textit{Python for everyone} eller online-material.
 \item 2 föreläsningar / vecka
 \item 3 handledda datorlabbstillfällen / vecka
 \item Föreläsningar hålls av:
 \begin{itemize}
  \item Mikael Öhman, postdoc på tillämpad mekanik \url{mikael.ohman@chalmers.se}
  \item Thomas Svedberg, C3SE (centrum för vetenskapliga och tekniska beräkningar)
 \end{itemize}
\end{itemize}
\end{frame}

\begin{frame}
\frametitle{DAT171 - Varför Python}
\begin{itemize}
 \item Kan köras interaktivt, enkel syntax, enkel debugging
 \item Många programmeringstekniker; objektorienterad, funktionell, imperativ
 \item Används som scriptspråk i många applikationer (e.g. Abaqus, Salome, Autodesk)
 \item Många bibliotek skrivna i andra språk har bindningar till Python (OpenFOAM, )
 \item Lämpligt för många program
 \begin{itemize}
 \item Applied mechanics (e.g. beräkningar)
 \item System control and mechatronics (e.g. robotar)
 \item Production engineering (e.g. simuleringar fabriker/logistik)
 \item Product development (e.g. automatisera CAD, databaser)
 \end{itemize}
 \item Alumnerankät frågar efter mer programmering
\end{itemize}
\end{frame}

\begin{frame}
\frametitle{DAT171 - Kursmål}
\begin{itemize}
 \item Lärandemål: \textit{ Efter avklarad kurs skall studenten självständigt kunna skriva objektorienterad mjukvara med Python.
 Vidare skall studenten kunna läsa referenslitteratur för programmeringsspråket och speciellt kunna använda Scipy/Numpy för numeriska beräkningar, och PySide (Qt) för att skriva grafiska användergränssnitt.}
 \item Innehåll:
 \begin{itemize}
  \item Grundläggande byggstenarna i Python (variabler, villkor, loopar, bibliotek, felsökning)
  \item Datastrukturer (träd, ordböcker, köer, tupler)
  \item Objektorienterad programmering (klasser, objekt, arv, polymorfism, abstrakta klasser)
  \item PySide för att skapa grafiska användargränssnitt
  \item NumPy (matriser, vektorer, linjär algebra)
  \item Scipy (paker för numeriska beräkningar)
  \item Matplotlib
  \item IPython (interactive python)
 \end{itemize}
\end{itemize}
\end{frame}

\begin{frame}
\frametitle{DAT171 - Examination}
\begin{itemize}
 \item 3 obligatoriska inlämningsuppgifter där ni kommer att få
 \begin{itemize}
 \item använda IPython + Numpy + Matplotlib $\approx$ Matlab-ersättare
 \item skriva egna datastrukturer med hjälp av klasser
 \item skriva egna bibliotek
 \item göra interaktiva program med grafiska användargränssnitt
 \end{itemize}
 \item Tentamen som betygsätter kursen
 \begin{itemize}
  \item ges i datorsal
  \item består av mindre uppgifter, där minst en uppgift har ett interaktivt GUI.
  \item tillåter referenslitteratur
 \end{itemize}
\end{itemize}
\end{frame}
\end{document}
