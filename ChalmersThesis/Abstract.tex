\begin{abstract}
It is important to study the aerodynamic effects on high speed trains, due to both comfort and stability.
The Swedish high speed trains are aiming to go at a speed of 250 km/h. The present work closely connects the aerodynamic effects with the vibration dynamics within the train. 

Two scenarios are simulated, two trains meeting each other and a train leaving a tunnel and is hit by a strong wind gust (35 m/s). From the aerodynamic part, computational fluid dynamics (CFD) is used with the $k$-$\zeta$-$f$ turbulence model. To simulate both scenarios a moving mesh needs to be used. From the CFD the moments and forces from the pressure and traction on the train body is calculated, and these loads are taking into a low order mathematical model that simulates the vibration dynamics in the train.

For the scenario with meeting trains the train experiences some slight vibrations, causing discomfort, but has no impact on the stability of the train, but for the scenario with the tunnel and side wind the train has a very high risk of derailment. 

From the results of the dynamics simulations a comfort and stability measurements were constructed based on the vibrations in the train car and the risk of wheel climbing. From simulating different speeds of the train it could be seen that the comfort and stability changes linearly with the speed.
Work was also done to see how much impact the coupling between the car bodies can have on the comfort and stability. A comparison was made to a simple, stiff coupling and one that optimizes a set of passive dampers. It's seen that the coupling can make a difference of around 8\% in comfort and stability, which equals an effect of lowering the speed 3-5 m/s.

Semi-active dampers of sky-hook and ground-hook type were also tested in the coupling, but they only showed insignificant changes or changes to the worse.
\end{abstract}