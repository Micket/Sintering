\documentclass{article}
\usepackage{fontspec} % For loading fonts
\defaultfontfeatures{Mapping=tex-text}


\usepackage{xunicode,url,parskip} % Formatting packages


\usepackage[usenames,dvipsnames]{xcolor} % Required for specifying custom colors

\usepackage{hyperref} % Required for adding links   and customizing them
\definecolor{linkcolour}{rgb}{0,0.2,0.6} % Link color
\hypersetup{colorlinks,breaklinks,urlcolor=linkcolour,linkcolor=linkcolour} % Set link colors throughout the document

\usepackage{titlesec} % Used to customize the \section command
\titleformat{\section}{\Large\scshape\raggedright}{}{0em}{}[\titlerule] % Text formatting of sections
\titlespacing{\section}{0pt}{3pt}{3pt} % Spacing around sections

\begin{document}

\textit{
1-3 pages where you introduce yourself present your qualifications and describe your future research plans.
− Previous research fields and main research results.
− Future goals and research focus. Is there any specific projects and research issues you are primarily interested in?
}


\section{Personal letter}
My research interests are in development of new methods in finite element analysis, e.g.\ multiscale simulations, particle FEM, isogeometric analysis, contact simulations, and XFEM.

I have worked as a very active developer of OOFEM (www.oofem.org) for the past 4 years, accumulating over 1300 changes over 350 thousand lines of code (which is more than all other developers combined).
Apart from developing the basis for my own research in multiscale simulations, my main focus during that time has been to gradually improve the overall design of the infrastructure in order to simplify and generalize the code, as well as bringing in a series of performance improvements.
During this time I have also worked as the coordinator of the OOFEM development between Chalmers an the Czech Technical University in Prague.
This often means recommending strategies for how to implement functionality in OOFEM, and acting as technical support.

With the large amount of time spent developing finite element code, teaching in numerous courses in the finite element method and material mechanics, I believe I possess the right expertise to allow me to drive the research field forward in the topic of polycrystalline materials.

\section{Future research goals}
The ever increasing complexity of numerical simulations poses major concerns for continued development.
I believe with this can be solved by introducing established software development practices such as
\begin{itemize}
 \item Modular design
 \item High level abstractions
 \item Extensive automatic testing
 \item Documentation
 \item Collaboration; Version control, code review
 \item Open source
\end{itemize}
To often, sources to implementations are not shared, not tested, mostly undocumented, and barely usable by anyone but the original author.
I believe this must change for our research to maximize its full potential.
The high standard we hold for the research articles should also apply to the implementations.


My hope is to have the opportunity to continue development in OOFEM, not only adding new features but also improving the core infrastructure in order to enable faster, easier, and safer programming environment for all developers.


\end{document}
