\documentclass{article}
\usepackage{fontspec} % For loading fonts
\defaultfontfeatures{Mapping=tex-text}


\usepackage{xunicode,url,parskip} % Formatting packages


\usepackage[usenames,dvipsnames]{xcolor} % Required for specifying custom colors

\usepackage{hyperref} % Required for adding links   and customizing them
\definecolor{linkcolour}{rgb}{0,0.2,0.6} % Link color
\hypersetup{colorlinks,breaklinks,urlcolor=linkcolour,linkcolor=linkcolour} % Set link colors throughout the document

\usepackage{fullpage}
\usepackage{titlesec} % Used to customize the \section command
\titleformat{\section}{\Large\scshape\raggedright}{}{0em}{}[\titlerule] % Text formatting of sections
\titlespacing{\section}{0pt}{3pt}{3pt} % Spacing around sections

\begin{document}

\begin{center}
\Large
 QUALIFICATIONS, Öhman, Ref 20140011
\end{center}


\section{Personal letter}
My research interests are in development of new methods in finite element analysis, e.g.\ multiscale simulations, particle FEM, isogeometric analysis, contact simulations, and XFEM.
I'm also very interested in software development, both in my research and in my spare time.

As a PhD student, my work has focused on the problem of sintering with a new formulation for dealing with incompressibility in multiscale simulations.
I have also been a very active developer of OOFEM (www.oofem.org) for the past 4 years, accumulating over 1300 changes with over 350 thousand modified lines of code (which, by comparison, is more than all other developers combined for that period).
Apart from developing the basis for my own research, my focus has been to gradually improve the overall design of the infrastructure in order to simplify and generalize the code, as well as bringing in a series of performance improvements.
During this time I have also worked as the coordinator of the OOFEM development between Chalmers an the Czech Technical University in Prague.
This often means recommending strategies for how to implement functionality in OOFEM, and acting as technical support for fellow researchers at Chalmers.
I've also done a few minor contributions to OOFEM on behalf of industrial partners.

With the large amount of time spent developing finite element code, teaching in numerous courses in the finite element method and material mechanics, I believe I possess the right expertise to allow me to drive the research field forward in this postdoctoral position on polycrystalline materials.

\section{Future research goals}
The ever increasing complexity of new numerical simulations poses major concerns for future research.
Too often, sources to implementations are not shared, not tested, mostly undocumented, and written almost from scratch by a single PhD student.
Not only is that an inefficient strategy, but it is also prone to bugs.
Perhaps even more important, it often makes the software so specialized and hard to figure out, that it is unusable for anyone but the authors themselves.
To reach out to industry and fellow researchers I believe this must change, and that the solution is to use established software development practices such as
\begin{itemize}
 \item Modular design
 \item High level abstractions
 \item Extensive automatic testing
 \item Documentation
 \item Collaboration; Version control, code review
 \item Open source
\end{itemize}
The high standard we hold for the research articles should also apply to the implementations.
%As Open Access and Open Research advances, it follows Open Source software.

My hope is to have the opportunity to continue development of a shared platform for research, not only adding features, but also core infrastructure in order to enable faster, easier, and safer programming environment for all developers and users.
Such a platform will also foster new collaborations, both with industry and academia, promoting the research internationally.

\vspace{1cm}\hspace{2cm}\textit{Mikael Öhman}

\end{document}
