%%%%%%%%%%%%%%%%%%%%%%%%%%%%%%%%%%%%%%%%%
% Plasmati Graduate CV
% LaTeX Template
% Version 1.0 (24/3/13)
%
% This template has been downloaded from:
% http://www.LaTeXTemplates.com
%
% Original author:
% Alessandro Plasmati (alessandro.plasmati@gmail.com)
%
% License:
% CC BY-NC-SA 3.0 (http://creativecommons.org/licenses/by-nc-sa/3.0/)
%
% Important note:
% This template needs to be compiled with XeLaTeX.
% The main document font is called Fontin and can be downloaded for free
% from here: http://www.exljbris.com/fontin.html
%
%%%%%%%%%%%%%%%%%%%%%%%%%%%%%%%%%%%%%%%%%

%----------------------------------------------------------------------------------------
%	PACKAGES AND OTHER DOCUMENT CONFIGURATIONS
%----------------------------------------------------------------------------------------

\documentclass[a4paper,10pt]{article} % Default font size and paper size
\usepackage[backend=biber,bibstyle=reading,mincitenames=2]{biblatex}
\usepackage{fontspec} % For loading fonts
\defaultfontfeatures{Mapping=tex-text}
% \setmainfont[SmallCapsFont = Fontin SmallCaps]{Fontin} % Main document font

\usepackage{xunicode,url,parskip} % Formatting packages

\usepackage[usenames,dvipsnames]{xcolor} % Required for specifying custom colors

% \usepackage{fullpage}
\usepackage[left=2cm, right=2cm, bottom=2cm, top=2cm]{geometry}
% \usepackage[big]{layaureo} % Margin formatting of the A4 page, an alternative to layaureo can be \usepackage{fullpage}
% To reduce the height of the top margin uncomment: \addtolength{\voffset}{-1.3cm}

\usepackage{hyperref} % Required for adding links	and customizing them
\definecolor{linkcolour}{rgb}{0,0.2,0.6} % Link color
\hypersetup{colorlinks,breaklinks,urlcolor=linkcolour,linkcolor=linkcolour} % Set link colors throughout the document

\usepackage{titlesec} % Used to customize the \section command
\titleformat{\section}{\Large\scshape\raggedright}{}{0em}{}[\titlerule] % Text formatting of sections
\titlespacing{\section}{0pt}{3pt}{3pt} % Spacing around sections

\addbibresource{publications.bib}

\begin{document}

\pagestyle{empty} % Removes page numbering

%\font\fb=''[cmr10]'' % Change the font of the \LaTeX command under the skills section

%----------------------------------------------------------------------------------------
%	NAME AND CONTACT INFORMATION
%----------------------------------------------------------------------------------------

% COVER LETTER:

Mikael Öhman
mobil: 073-6837674
epost: mikael.ohman@chalmers.se
Tjänst: Developer numerical algorithms
Referensnummer: REF-505

Hej Jacob!

Jag har sett er annons om utvecklare till COMSOL Multiphysics, och jag tycker att tjänsten verkar mycket intressant.
Jag har min doktorsexamen från Material- och beräkningsmekanik, Chalmers (2014) och fortsatte därefter med en Postdoc-tjänst på samma avdelning.

Jag är en programmerare in i själen som gillar att jobba i grupp för att tackla svåra tekniska problem.
Min erfarenhet inom utveckling av FE-lösare i C++ kan tydligas ses via mina stora bidrag till OOFEM (www.oofem.org) som jag gjort under min tid som doktorand och forskare.
Med över 2000 commits så har jag involverat mig i alla delar av koden, t.ex. strömningsproblem, värmeöverföring, XFEM, skalformuleringar, homogenisering, glesa matris-format, prestandaoptimering, parallellisering, m.m. Jag har varit support för många andra utvecklare, delvis på Chalmers, men även internationellt via OOFEMs forum (www.oofem.org/forum).
Som lärare på Chalmers har jag även undervisat i Java, Matlab, Python, FEM, hållfasthetslära och matematik.

Jag tror att min bakgrund passar väldigt bra in på de kunskaper och erfarenheter som efterfrågas. Jag har alltid tyckt att COMSOL sticker ut bland konkurrenterna då det ger ett väldigt modernt mjukvarupaket i en bransch som annars ofta präglas av föråldrad design och bristfälliga grafiska interface. Det vore mycket spännande att få möjligheten att arbeta med utveckling av en kod som COMSOL Multiphysics.

Med vänliga hälsningar, Mikael Öhman.

https://www.linkedin.com/in/mikael-%C3%B6hman-71308274
https://github.com/Micket/
http://stackexchange.com/users/154571/mikael-%C3%96hman

\par{{\centering
%\Large...,.
REF-505\\\bigskip
\Huge Mikael Öhman\bigskip
\par} % Your name

\section{Personal Data}

\begin{tabular}{rl}
\textsc{Place and Date of Birth:} & Sweden  | 19 April 1983 \\
\textsc{Address:} & Gibraltargatan 21C, Göteborg, Sweden \\
\textsc{Phone:} & 073 6837674\\
\textsc{email:} & \href{mailto:mikael.ohman@chalmers.se}{mikael.ohman@chalmers.se} 
                  \href{mailto:la.mikael.ohman@gmail.com}{la.mikael.ohman@gmail.com}
\end{tabular}

%----------------------------------------------------------------------------------------
%	EDUCATION
%----------------------------------------------------------------------------------------

\section{Education}

\begin{tabular}{rl}

%------------------------------------------------

\textsc{June} 2014 & PhD in \textsc{Solid and Structural Mechanics} \\
& \textbf{Chalmers University of Technology}, Göteborg\\
& \small Opponent: Prof. Julia \textsc{Mergheim}\\
&\\

%------------------------------------------------

\textsc{December} 2011 & Lic. of Eng. in \textsc{Solid and Structural Mechanics} \\
& \textbf{Chalmers University of Technology}, Göteborg\\
& Thesis: ``Computational Modeling of Liquid-Phase Sintering based on \\
& Homogenization''\\
& \small Discussion leader: Prof. Jörn \textsc{Mosler}\\
&\\

%------------------------------------------------

\textsc{April} 2009 & MSc in \textsc{Solid and Fluid mechanics}\\
& \textbf{Chalmers University of Technology}, Göteborg\\
& Specialization: Structural Dynamics | Major: Mechanical Engineering\\
& Thesis: ``Stability of High Speed Train under Aerodynamic Excitation''\\
& \small Advisor: Prof. Viktor \textsc{Berbyuk}\\
&\\

%------------------------------------------------

\textsc{June} 2007 & BSc in \textsc{Mechanical Engineering}\\
& \textbf{Chalmers University of Technology}, Göteborg\\
& Thesis: ``Modeling of a Wire Suspended Roof over Soccer Field''\\
& \small Examiner: Peter \textsc{Möller}\\
&\\

%------------------------------------------------
\end{tabular}

%----------------------------------------------------------------------------------------
%   WORK EXPERIENCE 
%----------------------------------------------------------------------------------------

\section{Work Experience}

\begin{tabular}{r|p{11cm}}
\textsc{June -- August 2008} & Assistant at division of Material \& Computational Mechanics, Chalmers University of Technology, Göteborg\\
& \footnotesize{Developed code for automatic detection of local curvature of contact surfaces in wheel-rail interaction for the Innotrack project. Automated finite element analysis of plastic contact problems.}\\
\multicolumn{2}{c}{} \\
\textsc{2006--2009} & Teaching assistant, Chalmers University of Technology, Göteborg\\
& \footnotesize{Courses listed subsequently.}\\
\multicolumn{2}{c}{} \\
\textsc{December 2013} & Freelance work, LMAT Ltd, United Kingdom\\
& \footnotesize{Developed code to perform finite element analysis of layered composites.}\\
\multicolumn{2}{c}{} \\
\textsc{June 2014 -- June 2016} & Postdoc, Chalmers University of Technology, Göteborg\\
& \footnotesize{Developing models for microstructural investigations on WC-Co hardmetals.
  Developed new course in object oriented programming in Python.}\\

\end{tabular}


\section{References}
Mikael Enelund \href{mailto:mikael.enelund@chalmers.se}{mikael.enelund@chalmers.se}, 
Kenneth Runesson \href{mailto:kenneth.runesson@chalmers.se}{kenneth.runesson@chalmers.se},\\
Fredrik Larsson \href{mailto:fredrik.larsson@chalmers.se}{fredrik.larsson@chalmers.se},
Magnus Ekh \href{mailto:magnus.ekh@chalmers.se}{magnus.ekh@chalmers.se}
%Peter Möller \href{mailto:peter.moller@chalmers.se}{peter.moller@chalmers.se}



%----------------------------------------------------------------------------------------
%   INTERESTS AND ACTIVITIES
%----------------------------------------------------------------------------------------
\newpage
\section{Computer and programming experience}

I'm proficient in many programming languages; C++, C, Python, Matlab, Java, Fortran, Haskell, PHP.
Primarily, I use modern C++ and Python and I have for the past 6 years been the most active developer of the open source, multi-physics, finite element solver OOFEM \url{www.oofem.org}. I have also had the responsibility of merging in changes from Chalmers with the main project repository, and my large contributions to user and developer support can be seen on the project forum \url{www.oofem.org/forum}.
My contributions to OOFEM span a large section of techniques in FEM, e.g.\ structural elements, material modeling, performance tuning, XFEM, constraints, solver techniques, computational homogenization.


My activity in open source projects can also be tracked on Github \url{www.github.com/Micket/}.

I'm very familiar with the related tools, such as
\begin{itemize}
 \item Compilers/interpreters, linkers, debuggers, static analyzers
 \item Build systems
 \item Version control
 \item Profilers
 \item API documentation
 \item IDEs
 \item Parallelization with OpenMP, MPI
\end{itemize}

I have over 10 years of experience using Linux full time, primarily Debian, OpenSUSE, RHEL.
I consider myself an expert in terminal usage and scripting.
% I have some experience in setting up services like Apache, FTP, SSH, and Docker for private use.


\section{List of peer-review journal publications}

\fullcite{ohman_computational_2012}

\fullcite{ohman_computational_2013}

\fullcite{ohman_computational_2015}

\section{Journal publications in submission and progress}

\fullcite{ohman_computational_2016}

\fullcite{johansson_ccbuilder:_2016}

\fullcite{ohman_residual_2016}

% \fullcite{ohman_voigt_2016}

\fullcite{larsson_generalized_2016}

% \section{Other publications}
% 
% \fullcite{ohman_ccbuilder_2015}
% 
% \fullcite{ohman_formulation_2013}
% 
% \fullcite{ohman_fe2_2012}
% 
% \fullcite{ohman_computational_2011}
% 
% \fullcite{ohman_multiscale_2010}
% 
% \fullcite{ohman_stability_2009}

%----------------------------------------------------------------------------------------
%	INTERESTS AND ACTIVITIES
%----------------------------------------------------------------------------------------

\section{Interests}

Technology, Art, Open-Source software, Programming, Mathematics, Physics, Linux, Video games

%----------------------------------------------------------------------------------------
%   Teaching
%----------------------------------------------------------------------------------------

\section{Teaching Experience}

\begin{tabular}{r|p{11cm}}
\textsc{2016} 
              & \emph{Object Oriented Programming in Python (BSc/MSc)}: \footnotesize{Lecturer, examiner}\\
\multicolumn{2}{c}{} \\

\textsc{2015} 
	      & \emph{Object Oriented Programming in Python (BSc/MSc)}: \footnotesize{Lecturer, examiner (developed new course)}\\
              & \emph{Supervised bachelor thesis group}\\
\multicolumn{2}{c}{} \\

\textsc{2014} 
              & \emph{Finite element method (BSc)}: \footnotesize{Tutoring computer labs, grading.}\\
              & \emph{Mechanics of solids (BSc)}: \footnotesize{Problem solving sessions, computer labs, grading.}\\
\multicolumn{2}{c}{} \\

\textsc{2013} 
              & \emph{Finite element method (BSc)}: \footnotesize{Tutoring computer labs, grading.}\\
              & \emph{Mechanics of solids (BSc)}: \footnotesize{Problem solving sessions, computer labs, grading.}\\
\multicolumn{2}{c}{} \\

%------------------------------------------------

\textsc{2012} 
              & \emph{Finite element method --- basics}: \footnotesize{Computer labs, grading.}\\
              & \emph{Supervised bachelor thesis group}\\
              & \emph{Statics \& solid of mechanics (BSc)}: \footnotesize{Computer labs, problem solving sessions, grading}\\
              & \emph{Mechanics of solids (BSc)}: \footnotesize{Computer labs, problem solving sessions, grading}\\
\multicolumn{2}{c}{} \\

%------------------------------------------------

\textsc{2011} 
              & \emph{Mechanics of solids (BSc)}: \footnotesize{Computer labs, correcting handins.}\\
              & \emph{Advanced Material Mechanics}: \footnotesize{Tutoring and constructing computer labs, problem solving sessions, grading}\\
\multicolumn{2}{c}{} \\

%------------------------------------------------

\textsc{2010} 
              & \emph{Statics \& solid of mechanics (BSc)}: \footnotesize{Computer labs, problem solving sessions, grading}\\
              & \emph{Advanced Material Mechanics}: \footnotesize{Tutoring and constructing computer labs, problem solving sessions, grading}\\
              & \emph{Finite element method (BSc)}: \footnotesize{Tutoring computer labs, grading.}\\
\multicolumn{2}{c}{} \\

%------------------------------------------------

\textsc{2009} 
              & \emph{Advanced Material Mechanics}: \footnotesize{Tutoring and constructing computer labs, problem solving sessions, grading}\\
              & \emph{Finite element method (BSc)}: \footnotesize{Tutoring computer labs, grading.}\\
              & \emph{Statics \& mechanics of solids (BSc)}: \footnotesize{Computer labs, problem solving sessions, grading}\\
\multicolumn{2}{c}{} \\

%------------------------------------------------

\textsc{2008} 
              & \emph{Statics \& mechanics of solids (BSc)}: \footnotesize{Computer labs, problem solving sessions, grading}\\
              & \emph{Mechanics of solids (BSc)}: \footnotesize{Computer labs, problem solving sessions, grading}\\
              & \emph{Dynamics (BSc)}: \footnotesize{Computer labs, problem solving sessions, grading}\\
              & \emph{Finite element method (BSc)}: \footnotesize{Computer labs}\\
              & \emph{Computer introduction (BSc)}: \footnotesize{Lectures and computer labs.}\\
\multicolumn{2}{c}{} \\

%------------------------------------------------

\textsc{2007} 
              & \emph{Programming in Matlab (BSc)}: \footnotesize{Computer labs.}\\
              & \emph{Statics \& Dynamics (BSc)}: \footnotesize{Problem solving sessions, computer labs}\\
              & \emph{Programming in Java (BSc)}: \footnotesize{Computer labs. Correcting handins.}\\
              & \emph{Mathematics (BSc)}: \footnotesize{Computer labs.}\\
              & \emph{Computer introduction (BSc)}: \footnotesize{Lectures and computer labs.}\\
\multicolumn{2}{c}{} \\

%------------------------------------------------

\textsc{2006} 
              & \emph{Programming in Java (BSc)}: \footnotesize{Computer labs.}\\
              & \emph{Programming in Java (BSc)}: \footnotesize{Computer labs, correcting handins}\\
\multicolumn{2}{c}{} \\

%------------------------------------------------
% 
% \textsc{2005} & \emph{Programming in Java (BSc)}: \footnotesize{Tutoring computer labs.}\\
% \multicolumn{2}{c}{} \\

%------------------------------------------------
\end{tabular}

During my studies at Chalmers, I also partook in the Supplemental Instruction (SI) initiative, and was helping students in the subjects of Mathematics, Mechanics of Solids, and Programming.
I have had positive feedback from the students.

%----------------------------------------------------------------------------------------
%	GRADE TABLES
%----------------------------------------------------------------------------------------
\newpage

\section{Courses and Grades from Chalmers University of Technology}
% \par{
% \centering\Large
% \hypertarget{grds}{Courses and Grades from Chalmers University of Technology}\par}

\begin{center}
\begin{tabular}{lcc}

\multicolumn{1}{c}{\textsc{BSc Courses}} & \textsc{Grade}\\
\hline
Material science and engineering, part A  & 3\\
Material science and engineering, part B & 3\\
Mechatronics & 4\\
Programming (Java) & 5\\
Industrial production \& organization & 3\\
Machine elements & 4\\
Integrated design and manufacturing & 4\\
Mathematical statistics & 4\\
Calculus in a single variable & 5\\
Introductory course in mathematics & 4\\
Calculus in several variables & 4\\
Linear algebra & 4\\
The finite element method & 4\\
Transforms and differential equations & 5\\
BSc thesis & 4\\

\multicolumn{1}{c}{\textsc{MSc Courses}} & \\
\hline
Rigid body dynamics & 4\\
Mechanics of solids and fluids & 4\\
Computations in mechanics & 4\\
Project in applied mechanics & 5\\
Applied structural dynamics & 5\\
Vibration control & 4\\
Advanced material mechanics & 5\\
Finite element method --- applications & 5\\
Material mechanics & 5\\
Fatigue design & 4\\
Fundamental structural dynamics & 5\\
History of science & 4\\

\multicolumn{1}{c}{\textsc{Voluntary courses}} & \\
\hline
Stochastic optimization algorithms & 3\\
Algebra & 4\\
Algorithms & 4\\
Data structures & 3\\
Advanced programming in C & 3\\
Introduction to functional programming & 5\\
Partial differential equations, first course & 3\\
High performance computing & 4\\
Markov theory & 4\\
Finite automata theory and formal languages & 4\\

\multicolumn{1}{c}{\textsc{PhD courses}} & \\
\hline
Functional analysis & \\
Finite element method --- solids & \\
Computational nonlinear mechanics & \\
Teaching learning \& evaluation & \\
Ethics, science \& society & 
\end{tabular}
\end{center}


%----------------------------------------------------------------------------------------

\end{document}
