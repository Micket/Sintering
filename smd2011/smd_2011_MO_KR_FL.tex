\documentclass{article}
\usepackage{anysize}
\marginsize{2.5cm}{1.8cm}{2.2cm}{2.5cm}
\usepackage[utf8]{inputenc}
\usepackage[T1]{fontenc}
\usepackage{graphicx}
\usepackage{url}
\pagestyle{empty}
\usepackage{color}

\begin{document}
\begin{center}
\section*{FE\textsuperscript{2} Modeling of Liquid-Phase Sintering}
%
\begin{minipage}[t]{\textwidth}
\centering
\underline{Mikael Öhman}\textsuperscript{1}, Kenneth Runesson\textsuperscript{2},  Fredrik Larsson\textsuperscript{3}\\
\vspace{0.5cm}
\textsuperscript{1}Tillämpad Mekanik, Chalmers Tekniska Högskola, Göteborg, email: mikael.ohman@chalmers.se\\
\textsuperscript{2}Tillämpad Mekanik, Chalmers Tekniska Högskola, Göteborg, email: kenneth.runesson@chalmers.se\\
\textsuperscript{3}Tillämpad Mekanik, Chalmers Tekniska Högskola, Göteborg, email: fredrik.larsson@chalmers.se
\end{minipage}
\end{center}

\large
% =================================================================================================
In this contribution, we discuss the two-scale modeling of liquid-phase sintering of hardmetal, which is composed of hard particles and a melted binder.
%we discuss the two-scale modeling of sintering of hardmetal, which is composed of hard particles (WC) and a melted binder (Co).
The sintering phenomenon can be explained on the mesoscale by surface tension, whose homogenized effect is the so-called sintering stress.
From the macroscopic perspective, the compacted specimen (green body) shrinks due to this volumetric sintering stress. In the case of inhomogeneous
initial density in the green body, the sintering process can result in unwanted distortions in the final product.

Computational homogenization has been applied to a Representative Volume Element (RVE) of the mesoscale structure.
A distinct advantage of this approach is that the complex behavior of sintering can be captured using relatively simple material models with
measurable parameters for each microconstituent. The deformation of melted binder is modeled as nonlinear Stokes' flow, driven by surface
tension (energy) on the free surface (interface between binder and porespace). Hard inclusions are modeled by a large viscosity.
%Different types of prolongation conditions (boundary conditions on the RVE) are possible. A quite novel type of ``weakly periodic velocity''
%is compared to Dirichlet and Neumann conditions. Although the latter conditions are classical for solid deformation, their application to
%the present problem is not entirely trivial.
It is important to accurately discern the (possibly) remaining porosity after finished sintering.
The need to simulate how the individual pores can vanish represents a particular challenge on the meshing strategy.

Numerical results are obtained for 2D, fully coupled (or nested), FE\textsuperscript{2} computations as a proof of concept.
For the mesocsale analysis, Taylor-Hood elements are adopted for the Stokes' flow, whereas quadratic edge elements are used to represent the surface tension.
The macroscale response changes character (from compressible to incompressible) when the porosity vanishes; hence, a mixed velocity-pressure format is a suitable choice that works in both regimes.
%Linear triangular elements are used for the macroscale velocity field. It is demonstrated how an inhomogeneous initial density may result in inhomogeneous shrinkage
%and remaining macroscopic distortion.

% =================================================================================================

\begin{thebibliography}{99}
\normalsize

\bibitem{Steinmann}
A. Javili, P. Steinmann.
A finite element framework for continua with boundary energies. Part I: The two-dimensional case,
\textit{Comput. Meth. Appl. Mech. Engng.},
198 pp. 2198--2208,
2009

\bibitem{Ohman}
M. Öhman, K. Runesson, F. Larsson.
Multiscale modeling of sintering of hardmetal,
\textit{23\textsuperscript{rd} Nordic Seminar on Computational Mechanics}, (A. Eriksson, G. Tibert, eds.), pp. 334--336, 2010.
\end{thebibliography}
\end{document}
% ================================================================================================= 

\end{document} 