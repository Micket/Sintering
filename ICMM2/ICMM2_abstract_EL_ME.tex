% =================================================================================================
% documentstyle
\documentclass[11pt,a4paper]{article}

% userpackages
\usepackage{times}
\usepackage[colorlinks=true,linkcolor=black,citecolor=black,urlcolor=black]{hyperref}

% set-up
\pagestyle{empty}
\setlength{\parindent}{0mm}
\setlength{\parskip}{2mm}

% =================================================================================================

\begin{document}
\begin{center}

% Title of the abstract
\textbf{\Large Multiscale modeling of the mechanical behaviour of pearlitic steel}
\\[5mm]

% Full names of all authors
\textrm{\large Mikael Öhman${}_{}^{1}$, Kenneth Runesson${}_{}^{1}$}
\\[5mm]

% Affiliations and addresses
\textrm{${}_{}^{1}$Department of Applied Mechanics\\
Chalmers University of Technology, Gothenburg, Sweden\\
e-mail: \url{mikael.ohman@chalmers.se}}
\\[5mm]
\end{center}

% =================================================================================================
In this contribution, we discuss the multiscale modeling of sintering of hard metal, which is composed of hard particles (WC) with a melted binder (Co).
The sintering phenomenon on the can be explained mesoscale by the effect of surface tension on the melted binder, and
the homogenized effect of the surface tension is the so called sintering stress.
From the macroscopic perspective, the specimen (green body) shrinks due to this volumetric sintering stress. In the case of inhomogeneous
initial density in the green body, this procedure can result in unwanted deformations.

%Basis:
%Sintering, sintering stress, inhomogeneous shrinkage

Finite elements and computational homogenization have been applied to a Representative Volume Element (RVE) of the mesoscale.
The melted binder is modeled as a Stokes flow, with surface tension potential energy on the free surface. Hard inclusions are modeled by applying a high viscosity.
%This is coupled to the macroscopic stress and strain and the mesoscale  is solved for every time step in every integration point.
Different choices of boundary conditions are available on the RVE for the prolongation.
The advantage of this kind of approach is that it can capture the complex behavior of sintering with simple material models with measurable parameters.
%For the macro scale
%Method:
%Multi scale (FE$^2$, Homogenization)
%Macro scale: deformation.
%Micro scale: Stokes flow, surface tension.

%Examples,
Numerical examples are evaluated for a 2D, fully coupled, FE$^2$ problem as a proof of concept.
Taylor-Hood elements for the linear Stokes flow, and quadratic edge elements for the surface tension, are employed for the mesoscale.
Dirichlet boundary conditions are used for the prolongation of the macroscopic strain rate.
Linear triangular elements with one integration point are used on the macroscopic scale.
This is shown for both with and without hard inclusions and a viscous melt subjected to surface tension.
The examples show how an inhomogeneous initial density will cause the macroscopic body to
undergo deformation
%shear
during the sintering process.
On the mesoscale, particles are merged, driven by the surface tension on the free pore surface.

% =================================================================================================

\begin{thebibliography}{99}

% article style 
% -----------------------------------------------
% \bibitem{art}
% Firstauthor, A. B., ..., 
% Article title,
% Full journal name,
% Volume number, 
% First page number, 
% Year.

\bibitem{Ekh}
Ekh, M.  et al.,
A model framework for anisotropic damage coupled to crystal (visco)plasticity,
International Journal of Plasticity,
20,
2143,
2004
%
% conference proceedings
% -----------------------------------------------
% \bibitem{cp}
% Firstauthor, A. B., ..., 
% Contribution title,
% Proceedings name, 
% Editor names, 
% Volume number,
% First page number, 
% Year.

% book
% -----------------------------------------------
% \bibitem{book}
% Firstauthor, A. B., ..., 
% Book title,
% Publisher name, 
% Year.

\end{thebibliography}
\end{document}
% =================================================================================================