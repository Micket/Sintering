% =================================================================================================
% documentstyle
\documentclass[11pt,a4paper]{article}

% userpackages
\usepackage{times}
\usepackage[colorlinks=true,linkcolor=black,citecolor=black,urlcolor=black]{hyperref}

% set-up
\pagestyle{empty}
\setlength{\parindent}{0mm}
\setlength{\parskip}{2mm}

% =================================================================================================

\begin{document}
\begin{center}

% Title of the abstract
\textbf{\Large Multiscale modeling of the mechanical behaviour of pearlitic steel}
\\[5mm]

% Full names of all authors
\textrm{\large Erik Lindfeldt${}_{}^{1}$, Magnus Ekh${}_{}^{1}$}
\\[5mm]

% Affiliations and addresses
\textrm{${}_{}^{1}$Department of Applied Mechanics\\
Chalmers University of Technology, Gothenburg, Sweden\\
e-mail: \url{erik.lindfeldt@chalmers.se}}
\\[5mm]
\end{center}

% =================================================================================================

Pearlitic steel consists of cementite lamellae embedded in a ferritic matrix. These lamellae are
arranged in colonies within which the cementite orientation is (ideally) constant. In this
contribution a representative microscale model that captures the behaviour of the cementite and the
ferrite but also the interaction between these phases is proposed. In the microscale model the
ferrite is modelled by using crystal plasticity \cite{Ekh} while the cementite is modelled as
elastic.

The homogenised response from the micromodel is passed via the mesoscopic model, using a multiscale
modeling approach, to the macroscale model which represents the structural level. The mesoscale
model consists of nodules, with different crystallographic orientations of the ferrite, within which
different colonies reside.

Different types of prolongation conditions on the micromodel from the mesoscale deformation gradient
are discussed and and their results are compared. Furthermore, the number of directions of the
cementite lamellae and the number of crystallographic directions of the ferrite needed to obtain a
representative mesoscale behavior is investigated. 

Finally, numerical results for different orientation distributions of the
cementite lamellae that give different degree of anisoptropy of the pearlite are given and compared
with experimental results in a qualitative fashion. 

% =================================================================================================

\begin{thebibliography}{99}

% article style 
% -----------------------------------------------
% \bibitem{art}
% Firstauthor, A. B., ..., 
% Article title,
% Full journal name,
% Volume number, 
% First page number, 
% Year.

\bibitem{Ekh}
Ekh, M.  et al.,
A model framework for anisotropic damage coupled to crystal (visco)plasticity,
International Journal of Plasticity,
20,
2143,
2004
%
\bibitem{perlit}
Elwazri, A.M. et al.,
The effect of microstructural characteristics of pearlite on the mechanical properties of
hypereutectoid steel,
Materials Science and Engineering: A
404,
91,
2005

% conference proceedings
% -----------------------------------------------
% \bibitem{cp}
% Firstauthor, A. B., ..., 
% Contribution title,
% Proceedings name, 
% Editor names, 
% Volume number,
% First page number, 
% Year.

% book
% -----------------------------------------------
% \bibitem{book}
% Firstauthor, A. B., ..., 
% Book title,
% Publisher name, 
% Year.

\end{thebibliography}
\end{document}
% =================================================================================================