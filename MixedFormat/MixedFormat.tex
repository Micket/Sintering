\documentclass[a4paper,11pt]{article}
\usepackage[english]{babel}
\usepackage{csquotes}
\usepackage[margin=2cm,includehead,includefoot]{geometry}

\usepackage{amsmath,contmech}
\renewcommand{\ta}[1]{\mathbfit{#1}}
\renewcommand{\ts}[1]{\mathbfit{#1}}
\renewcommand{\td}[1]{\mathbfcal{#1}}
\renewcommand{\tf}[1]{\mathbfsfup{#1}}
\renewcommand{\Box}{\mdlgwhtsquare}
\renewcommand{\leadsto}{\rightsquigarrow}

\usepackage{fontspec}
\usepackage{unicode-math}
\usepackage{microtype}

\setmainfont[Ligatures=TeX]{xits}
\setmathfont[math-style=ISO]{xits-math}
%\setmainfont[Ligatures=TeX]{Latin Modern Roman}
%\setmathfont[math-style=ISO]{lmmath-regular.otf}

\usepackage{fancyhdr}
\headheight=14.5pt
\pagestyle{fancy}
\fancyhead[r]{\today}
\fancyhead[l]{Mikael Öhman, Fredrik Larsson}

\newcommand{\ATS}{\bar{\tf E}}
\newcommand{\BC}{\mathrm{BC}}

\usepackage{enumitem}
\setlist[itemize]{itemindent=-1.5em}
\begin{document}
Document composed after discussion with Fredrik on how to split to a mixed macroscale formulation.
On the macro level:
 \begin{gather}
  -\bar{\ts\sigma}_\dev + \diff \bar p = \ta b\\
  \diff\cdot \bar{\ta v} + \bar d_\vol = 0
 \end{gather}

And the extension to the constitutive driver:
 \begin{gather}
  [\bar{\ts\sigma}_\dev, \bar{d}_\vol; \ATS_\dev] = f(\bar{\ts d}_\dev,\bar p)
 \end{gather}

If dense RVE $\implies$
 \begin{gather}
  \bar d_\vol = 0\\
  \ts d = \ts d_\dev \leadsto \bar{\ts\sigma}_\dev, \; \ATS_\dev
 \end{gather}
else
 \begin{gather}
  \bar d_\vol^{(0)} = 0\\
  \bar{\ts d}^{(k)} = \bar{\ts d}_\dev + \frac13 \bar d_\vol^{(k)} \ts I\\
  \bar p \approx \bar p^{(k)} + E_\eqv \Delta d_\vol\\
  \bar d_\vol^{(k+1)} = \bar d_\vol^{(k)} - E_\eqv^{-1} (\bar p^{(k)} - \bar p)
 \end{gather}

To perform Newton iterations the tangent $E_\eqv$ is needed;
 \begin{gather}
  \dif\bar{\ts\sigma} = \ATS \dprod \dif\bar{\ts d} \implies \\
  \dif\bar{\ts\sigma}\dprod \ts I = \dif\bar{\ts\sigma}_\dev - \dif p \ts I \dprod \ts I 
    = \ts I \dprod \ATS \dprod \dif \bar{\ts d}_\dev + \ts I \dprod \ATS \dprod \ts I \frac13 \dif \bar d_\vol \implies\\
  \dif p = \underbrace{- \frac19\ts I \dprod \ATS \dprod \ts I}_{=E_\eqv} \dif \bar d_\vol 
 \end{gather}

An possible alternative approach would be to mix the boundary conditions
 \begin{gather}
  \ta v = \bar{\ts d}_\dev\cdot(\ta x - \bar{\ta x}) + \frac13 \bar{d}_\vol (\ta x-\bar{\ta x})\text{ on } \Gamma_\Box \\
  \ta t^+ = -(\bar{p}-\bar{p}_0) \ta n \text{ on } \Gamma_\Box
 \end{gather}
where $\bar d_\vol$ is unknown and $\bar{p}_0$ is the ``internal pressure'' (assuming no pores on the boundary)


\section{The macro-scale problem}
Consider the macroscale problem at quasistatic equilibrium, stated as
\begin{align}
 -\bar{\ts\sigma}\cdot \diff = \bar{\ts b} \quad \text{in } \Omega
\end{align}
where we seek the macroscale velocity $\bar{\ts v}$. In particular, we identify two possible responses (induced by the submodel presented below):
\begin{gather}
 \bar{\ts\sigma} = \bar{\ts\sigma}_\Box(\bar{\ts d}) \label{eq:macro_compressible}\\
 \begin{aligned}
   \bar{\ts\sigma} &= \bar{\ts\sigma}_{\Box,\dev}(\bar{\ts d}_\dev, \bar p) - \bar p \ts I\\
   \bar{d}_\vol &= 0
 \end{aligned} \label{eq:macro_incompressible}
\end{gather}

where \eqref{eq:macro_compressible} pertains to compressible and \eqref{eq:macro_incompressible} to incompressible response.
In order to comply with both cases, we generalize the formulation as follows:
\begin{gather}
 \bar{\ts\sigma} = \bar{\ts\sigma}_{\Box,\dev}(\bar{\ts d}_\dev,\bar p) - \bar p\ts I\\
 \bar{d}_\vol - \bar{d}_{\Box,\vol}(\bar{\ts d}_\dev, \bar p) = 0
\end{gather}
Here we introduced $\bar{\ts d} = \left[ \bar{\ta v} \outerp \diff \right]^\sym$, $\bar{d}_\vol = \ts I \dprod \bar{\ts d}$ and $\bar{\ts d}_\dev = \bar{\ts d} - \frac{\bar d_\vol}{3}\ts I$ 

We shall now show that the formulation in ... comprises both cases in ..
First we note that case .. in eqv is trivially identified by 
\begin{align}
 \bar{\ts\sigma}_{\Box,\dev}(\bar{\ts d}_\dev, \bar p), \quad \bar{d}_{\Box,\vol}(\bar{\ts d}_\dev,\bar p) = 0
\end{align}
Case .. can be recast into .. by defining
\begin{align}
 \bar d_{\Box,\vol}(\bar{\ts d}_\dev, \bar p) = d^*,\quad \text{s.t. } -\frac13 \ts I\dprod \bar{\ts\sigma}_\Box(\bar{\ts d}_\dev, d^*) = \bar p.
\end{align}
Consequently, the deviatoric stress is defined as
\begin{align}
 \bar{\ts\sigma}_{\Box,\dev}(\bar{\ts d}_\dev,\bar p) = \bar{\ts\sigma}_\Box\left(\bar{\ts d}_\dev + \frac13 \bar{d}_{\Box,\vol}(\bar{\ts d}_\dev,\bar p)\ts I\right) + \bar p\ts I.
\end{align}

\subsection{Macro-scale problem}
Starting from the formulation in .. we state the mean form as that of finding $(\bar{\ta v},\bar p) \in \bar{\set V} \times \bar{\set{P}}$ s.t. 
\begin{align}
 \bar a\{\bar{\ta v},\bar p, \delta\bar{\ta v}\} - \bar b\{\bar p, \delta\bar{\ta v}\} &= 0   \quad \forall\; \delta\bar{\ta v} \in \bar{\set{V}}^0\\
 \bar b\{\delta\bar p, \bar{\ta v}\} - \bar c\{\bar{\ta v}, \bar p, \delta\bar p\}&= 0   \quad \forall\; \delta\bar p \in \bar{\set{P}}^0
\end{align}
in the ??? form above, we defined
\begin{align}
 \bar a\{\bar{\ta v},\bar p, \delta\bar{\ta v}\} &\defeq \int_\Omega \bar{\ts\sigma}_{\Box,\dev}([\bar{\ta v}\outerp \diff]^\sym,\bar p)\dprod \bar{\ta v}\outerp \diff]^\sym \dif V \\
 \bar b\{\delta\bar p, \bar{\ta v}\}             &\defeq \int_\Omega \bar p(\bar{\ta v}\cdot\diff)\dif V \\
 \bar c\{\bar{\ta v}, \bar p, \delta\bar p\}     &\defeq \int_\Omega \bar d_{\Box,\vol}([\bar{\ta v}\outerp \diff]^\sym,\bar p)\delta\bar p\dif V
\end{align}
Newton iterations are obtained as
\begin{align}
 \bar{a}'_p\{\bullet; \delta\bar{\ta v},\dif\bar{p}\} + \bar{a}'_v\{\bullet; \delta\bar{\ta v},\Delta\bar{\ta v}\} - \bar{b}\{\dif\bar p, \delta{\ta v}\} &= -\bar{a}\{\bullet; \delta\bar{\ta v}\} + \bar{b}\{\bullet, \delta\bar{\ta v}\}\quad \forall \delta\bar{\ta v}\\
\bar{b}\{\delta\bar p, \Delta\bar{\ta v}\} - \bar{c}'_v\{\bullet; \delta\bar{p}, \Delta\bar{ \ta v}\} &= -\bar{b}\{\delta\bar{p},\bullet\} + \bar{c}\{\bullet;\delta\bar{p}\}\quad \forall \delta\bar{p}
\end{align}
where
\begin{align}
 \bar{a}'_v &= \int_\Omega [\delta\bar{\ta v}\outerp\diff]^\sym_\dev \dprod \pd{\bar{\ts\sigma}_\dev}{\bar{\ts d}_\dev} \dprod [\Delta\bar{\ta v}\outerp\diff]^\sym_\dev \dif V\\
 \bar{b}'_p &= \int_\Omega [\delta\bar{\ta v}\outerp\diff]^\sym_\dev \dprod \pd{\bar{\ts\sigma}}{\bar p} \Delta\bar{p} \dif V\\
 \bar{c}'_v &= \int_\Omega \delta\bar{p} \pd{\bar{d}_{\Box,\vol}}{\bar{\ts d}_\dev} \dprod [\Delta\bar{\ta v}\outerp\diff]^\sym_\dev \dif V\\
 \bar{c}'_p &= \int_\Omega \delta\bar{p} \pd{\bar{d}_{\Box,\vol}}{\bar p} \Delta\bar p\dif V
\end{align}

\section{Nested analysis}
Assuming that the subscale problem is stated either as that of case (i) or case (ii) in \eqref{..} or \eqref{..} respectively.
\subsection{Compressible subproblem}
We assume that the subscale problem defines the function 
\begin{align}
 \bar{\ts\sigma} = \bar{\ts\sigma}_\Box(\bar{\ts d})\quad \text{and its derivative } \bar{\tf E} = \od{\bar{\ts\sigma}_\Box}{\bar{\ts d}}.
\end{align}
In order to utilize the general formulation in .., we formulate the intermediate problem of finding $\bar{d}_\vol^*$ s.t.
\begin{align}
 \frac13 \ts I\dprod \bar{\ts\sigma}_\Box\left(\bar{\ts d}_\dev + \frac13 \bar{d}_\vol^* \ts I\right) = -\bar p.
\end{align}
The iterative solution can be obtained by the update
\begin{align}
 \Delta\bar{d}_\vol &= \frac{1}{\bar K(\bar{\ts d}^{(k)})} \left[-\bar p - \frac13 \ts I\dprod \bar{\ts\sigma}_\Box\left(\bar{\ts d}^{(k)}\right)\right] \quad \text{with}\quad \bar{\ts d}^{(k)} = \bar{\ts d}_\dev + \frac13 \bar{d}_\vol^{(k)} \ts I\\
 \bar{d}_\vol^{(k+1)} &= \bar{d}_\vol^{(k)} + \Delta\bar{d}_\vol
\end{align}
Here, we introduced the bulk stiffness
\begin{align}
  \bar K(\bar{\ts d}) \defeq \frac19  \ts I\dprod \bar{\tf E} \dprod \ts I.
\end{align}
Finally, we state the equations pertinent to ... as follows
\begin{align}
 \left\{ \begin{aligned}
         \bar{\ts\sigma}_{\Box,\dev} &= \tf I_\dev \dprod \bar{\ts\sigma}_\Box(\bar{\ts d}_\dev + \frac13 \bar{d}_\vol^* \ts I)\\
         \bar{d}_{\Box,\vol} &= \bar{d}_\vol^*
        \end{aligned}
 \right.
\end{align}
Pertubations of input, i.e. $\bar{\ts d}_\dev$ and $\bar p$,  give rise to the problems:
\begin{align}
 \dif\bar{\ts\sigma}_{\Box,\dev} &= \tf I_\dev \dprod \dif\bar{\ts\sigma} = \tf I_\dev \dprod \bar{\tf E} \dprod (\dif\bar{\ts d}_\dev + \frac13 \ts I \dif\bar{d}_\vol) = \\
 \dif\bar{d}_{\Box,\vol} &= \od{\bar{d}_\vol^*}{\bar{\ts d}_\dev}\dprod \dif \bar{\ts d}_\dev + \od{\bar{\ts d}_\vol}{\bar p}\dif \bar p
\end{align}




 
\end{document}
