%% BioMed_Central_Tex_Template_v1.06
%%                                      %
%  bmc_article.tex            ver: 1.06 %
%                                       %

%%IMPORTANT: do not delete the first line of this template
%%It must be present to enable the BMC Submission system to
%%recognise this template!!

%%%%%%%%%%%%%%%%%%%%%%%%%%%%%%%%%%%%%%%%%
%%                                     %%
%%  LaTeX template for BioMed Central  %%
%%     journal article submissions     %%
%%                                     %%
%%          <8 June 2012>              %%
%%                                     %%
%%                                     %%
%%%%%%%%%%%%%%%%%%%%%%%%%%%%%%%%%%%%%%%%%


%%%%%%%%%%%%%%%%%%%%%%%%%%%%%%%%%%%%%%%%%%%%%%%%%%%%%%%%%%%%%%%%%%%%%
%%                                                                 %%
%% For instructions on how to fill out this Tex template           %%
%% document please refer to Readme.html and the instructions for   %%
%% authors page on the biomed central website                      %%
%% http://www.biomedcentral.com/info/authors/                      %%
%%                                                                 %%
%% Please do not use \input{...} to include other tex files.       %%
%% Submit your LaTeX manuscript as one .tex document.              %%
%%                                                                 %%
%% All additional figures and files should be attached             %%
%% separately and not embedded in the \TeX\ document itself.       %%
%%                                                                 %%
%% BioMed Central currently use the MikTex distribution of         %%
%% TeX for Windows) of TeX and LaTeX.  This is available from      %%
%% http://www.miktex.org                                           %%
%%                                                                 %%
%%%%%%%%%%%%%%%%%%%%%%%%%%%%%%%%%%%%%%%%%%%%%%%%%%%%%%%%%%%%%%%%%%%%%

%%% additional documentclass options:
%  [doublespacing]
%  [linenumbers]   - put the line numbers on margins

%%% loading packages, author definitions

%\documentclass[twocolumn]{bmcart}% uncomment this for twocolumn layout and comment line below
\documentclass{bmcart}

%%% Load packages
%\usepackage{amsthm,amsmath}
%\RequirePackage{natbib}
%\RequirePackage{hyperref}
\usepackage[utf8]{inputenc} %unicode support
%\usepackage[applemac]{inputenc} %applemac support if unicode package fails
%\usepackage[latin1]{inputenc} %UNIX support if unicode package fails

\usepackage{microtype}
\usepackage{amsmath, amssymb, mathtools, contmech}
\usepackage{graphicx, float}
\usepackage{microtype}
%\usepackage{tikz, pgfplots}
%\usetikzlibrary{arrows}
%\pgfplotsset{compat=1.9}
%\usetikzlibrary{external}
%\newcommand{\tikzinput}[1]{\input{#1.tikz}}
%\tikzexternalize

\newcommand{\tikzsetnextfilename}[1]{}
\newcommand{\tikzinput}[1]{\includegraphics{#1}}


\renewcommand{\doiurl}[1]{#1}

\newcommand{\figref}[1]{Figure~\ref{#1}}
\newcommand{\secref}[1]{Section~\ref{#1}}
\newcommand{\appref}[1]{Appendix~\ref{#1}}
\newcommand{\eqtrefrange}[2]{\eqref{#1}--\eqref{#2}}
\newcommand{\eqtref}[2]{\eqref{#1}}

%%%%%%%%%%%%%%%%%%%%%%%%%%%%%%%%%%%%%%%%%%%%%%%%%
%%                                             %%
%%  If you wish to display your graphics for   %%
%%  your own use using includegraphic or       %%
%%  includegraphics, then comment out the      %%
%%  following two lines of code.               %%
%%  NB: These line *must* be included when     %%
%%  submitting to BMC.                         %%
%%  All figure files must be submitted as      %%
%%  separate graphics through the BMC          %%
%%  submission process, not included in the    %%
%%  submitted article.                         %%
%%                                             %%
%%%%%%%%%%%%%%%%%%%%%%%%%%%%%%%%%%%%%%%%%%%%%%%%%


%\def\includegraphic{}
%\def\includegraphics{}


%%% Put your definitions there:
\startlocaldefs

% More specialized commands;
\DeclarePairedDelimiter{\homgen}{\langle}{\rangle_\rve}
\DeclarePairedDelimiter{\jmp}{[\![}{]\!]}
\newcommand{\prescribed}{\mathrm{pre}}
\newcommand{\on}{\quad\text{ on }}
\renewcommand{\dev}{\mathrm{d}}
\renewcommand{\vol}{\mathrm{v}}
\newcommand{\per}{\mathrm{per}}
\newcommand{\volume}{|\Omega_\rve|}
\newcommand{\ded}{\mathrm{d}}
\newcommand{\dep}{\mathrm{p}}
\newcommand{\Periodic}{\mathrm{P}}
\newcommand{\epspargs}{\{{\bar{\ts\epsilon}}_\dev, \bar{p}\}}
% Reduce the size of the rve box a bit:
\newcommand{\rve}{
  {\mathchoice
   {\mbox{\scalebox{0.67}{$\Box$}}}
   {\mbox{\scalebox{0.67}{$\Box$}}}
   {\mbox{\scalebox{0.5}{$\Box$}}}
   {\mbox{\scalebox{0.375}{$\Box$}}}
  }
}

\endlocaldefs


%%% Begin ...
\begin{document}

%%% Start of article front matter
\begin{frontmatter}

\begin{fmbox}
\dochead{Research}

%%%%%%%%%%%%%%%%%%%%%%%%%%%%%%%%%%%%%%%%%%%%%%
%%                                          %%
%% Enter the title of your article here     %%
%%                                          %%
%%%%%%%%%%%%%%%%%%%%%%%%%%%%%%%%%%%%%%%%%%%%%%

\title{On the Variationally Consistent Computational Homogenization of Elasticity in the Incompressible Limit}

%%%%%%%%%%%%%%%%%%%%%%%%%%%%%%%%%%%%%%%%%%%%%%
%%                                          %%
%% Enter the authors here                   %%
%%                                          %%
%% Specify information, if available,       %%
%% in the form:                             %%
%%   <key>={<id1>,<id2>}                    %%
%%   <key>=                                 %%
%% Comment or delete the keys which are     %%
%% not used. Repeat \author command as much %%
%% as required.                             %%
%%                                          %%
%%%%%%%%%%%%%%%%%%%%%%%%%%%%%%%%%%%%%%%%%%%%%%

\author[
   addressref={aff1},                   % id's of addresses, e.g. {aff1,aff2}
   %corref={aff1},                       % id of corresponding address, if any
   noteref={n1},                        % id's of article notes, if any
   email={mikael.ohman@chalmers.se}   % email address
]{\inits{MÖ}\fnm{Mikael} \snm{Öhman}}
\author[
   addressref={aff1},
   noteref={n1},                        % id's of article notes, if any
   email={kenneth.runesson@chalmers.se}
]{\inits{KR}\fnm{Kenneth} \snm{Runesson}}
\author[
   addressref={aff1},
   noteref={n1},                        % id's of article notes, if any
   email={fredrik.larsson@chalmers.se}
]{\inits{FL}\fnm{Fredrik} \snm{Larsson}}

%%%%%%%%%%%%%%%%%%%%%%%%%%%%%%%%%%%%%%%%%%%%%%
%%                                          %%
%% Enter the authors' addresses here        %%
%%                                          %%
%% Repeat \address commands as much as      %%
%% required.                                %%
%%                                          %%
%%%%%%%%%%%%%%%%%%%%%%%%%%%%%%%%%%%%%%%%%%%%%%

\address[id=aff1]{%
  \orgname{Department of Applied Mechanics, Chalmers University of Technology},
  \street{Hörsalsvägen 7},
  \postcode{41258}
  \city{Göteborg},
  \cny{Sweden}
}

%%%%%%%%%%%%%%%%%%%%%%%%%%%%%%%%%%%%%%%%%%%%%%
%%                                          %%
%% Enter short notes here                   %%
%%                                          %%
%% Short notes will be after addresses      %%
%% on first page.                           %%
%%                                          %%
%%%%%%%%%%%%%%%%%%%%%%%%%%%%%%%%%%%%%%%%%%%%%%

\begin{artnotes}
%\note{Sample of title note}     % note to the article
\note[id=n1]{Equal contributor} % note, connected to author
\end{artnotes}

\end{fmbox}% comment this for two column layout

%%%%%%%%%%%%%%%%%%%%%%%%%%%%%%%%%%%%%%%%%%%%%%
%%                                          %%
%% The Abstract begins here                 %%
%%                                          %%
%% Please refer to the Instructions for     %%
%% authors on http://www.biomedcentral.com  %%
%% and include the section headings         %%
%% accordingly for your article type.       %%
%%                                          %%
%%%%%%%%%%%%%%%%%%%%%%%%%%%%%%%%%%%%%%%%%%%%%%

\begin{abstractbox}

\begin{abstract} % abstract
The computational framework for Variationally Consistent Homogenization (VCH) of (near) incompressible solids is discussed.
To focus on the important issues, a model problem is considered for a composite whose constituents are characterized by nonlinear (hyper)elasticity and linear kinematics.
A canonical formulation of the subscale problem, pertinent to a Representative Volume Element (RVE), is established, whereby complete macroscale incompressibility is obtained straightforwardly as the limit situation when all constituents are incompressible.
The framework is sufficiently general to allow for the classical boundary conditions on the RVE as well as the generalized situation of weakly periodic boundary conditions.
Numerical results, demonstrating the seamless character of the computational algorithm at the fully incompressible, limit conclude the paper.
% \parttitle{First part title} %if any
% Text for this section.
% 
% \parttitle{Second part title} %if any
% Text for this section.
\end{abstract}

%%%%%%%%%%%%%%%%%%%%%%%%%%%%%%%%%%%%%%%%%%%%%%
%%                                          %%
%% The keywords begin here                  %%
%%                                          %%
%% Put each keyword in separate \kwd{}.     %%
%%                                          %%
%%%%%%%%%%%%%%%%%%%%%%%%%%%%%%%%%%%%%%%%%%%%%%

\begin{keyword}
\kwd{Multiscale}
\kwd{Computational homogenization}
\kwd{Incompressibility}
\kwd{Mixed variational formulations}
\end{keyword}

% MSC classifications codes, if any
%\begin{keyword}[class=AMS]
%\kwd[Primary ]{}
%\kwd{}
%\kwd[; secondary ]{}
%\end{keyword}

\end{abstractbox}
%
%\end{fmbox}% uncomment this for twcolumn layout

\end{frontmatter}

%%%%%%%%%%%%%%%%%%%%%%%%%%%%%%%%%%%%%%%%%%%%%%
%%                                          %%
%% The Main Body begins here                %%
%%                                          %%
%% Please refer to the instructions for     %%
%% authors on:                              %%
%% http://www.biomedcentral.com/info/authors%%
%% and include the section headings         %%
%% accordingly for your article type.       %%
%%                                          %%
%% See the Results and Discussion section   %%
%% for details on how to create sub-sections%%
%%                                          %%
%% use \cite{...} to cite references        %%
%%  \cite{koon} and                         %%
%%  \cite{oreg,khar,zvai,xjon,schn,pond}    %%
%%  \nocite{smith,marg,hunn,advi,koha,mouse}%%
%%                                          %%
%%%%%%%%%%%%%%%%%%%%%%%%%%%%%%%%%%%%%%%%%%%%%%

%%%%%%%%%%%%%%%%%%%%%%%%% start of article main body
% <put your article body there>

%%%%%%%%%%%%%%%%
%% Background %%
%%
\input{Contents}


%%%%%%%%%%%%%%%%%%%%%%%%%%%%%%%%%%%%%%%%%%%%%%
%%                                          %%
%% Backmatter begins here                   %%
%%                                          %%
%%%%%%%%%%%%%%%%%%%%%%%%%%%%%%%%%%%%%%%%%%%%%%

\begin{backmatter}

\section*{Abbreviations}
  \begin{itemize}
   \item RVE --- Representative Volume Element
   \item SVE --- Statistical Volume Element
   \item VCH --- Variationally Consistent Homogenization
   \item VMS --- Variational MultiScale
   \item VCMC --- Variationally Consistent Macrohomogeneity Condition
  \end{itemize}


\section*{Competing interests}
  The authors declare that they have no competing interests.

\section*{Author's contributions}
  The theory and manuscript was a joint work by all authors. MÖ developed the code for the numerical simulations.

\section*{Acknowledgements}
  The work was funded by the Swedish Research Council.

%%%%%%%%%%%%%%%%%%%%%%%%%%%%%%%%%%%%%%%%%%%%%%%%%%%%%%%%%%%%%
%%                  The Bibliography                       %%
%%                                                         %%
%%  Bmc_mathpys.bst  will be used to                       %%
%%  create a .BBL file for submission.                     %%
%%  After submission of the .TEX file,                     %%
%%  you will be prompted to submit your .BBL file.         %%
%%                                                         %%
%%                                                         %%
%%  Note that the displayed Bibliography will not          %%
%%  necessarily be rendered by Latex exactly as specified  %%
%%  in the online Instructions for Authors.                %%
%%                                                         %%
%%%%%%%%%%%%%%%%%%%%%%%%%%%%%%%%%%%%%%%%%%%%%%%%%%%%%%%%%%%%%

% if your bibliography is in bibtex format, use those commands:
\bibliographystyle{bmc-mathphys} % Style BST file
%\bibliography{bmc_article}      % Bibliography file (usually '*.bib' )
\bibliography{BoundaryPotential,Boundary_representation,FEM_Software,Mesh,Multiscale,Sintering}      % Bibliography file (usually '*.bib' )

% or include bibliography directly:
% \begin{thebibliography}
% \bibitem{b1}
% \end{thebibliography}

%%%%%%%%%%%%%%%%%%%%%%%%%%%%%%%%%%%
%%                               %%
%% Figures                       %%
%%                               %%
%% NB: this is for captions and  %%
%% Titles. All graphics must be  %%
%% submitted separately and NOT  %%
%% included in the Tex document  %%
%%                               %%
%%%%%%%%%%%%%%%%%%%%%%%%%%%%%%%%%%%

\end{backmatter}
\end{document}
%%
%% Do not use \listoffigures as most will included as separate files
% 
\section*{Figures}
\begin{figure}[h!]
\centering
\includegraphics{swisscheese}
\caption{\csentence{Generic micro-heterogeneous material consisting of inclusions in matrix (example)}\label{fig:swisscheese}}
\end{figure}

\begin{figure}[htb]
\centering
\includegraphics{swisscheese_periodic}
\caption{\csentence{RVE in 2D with ``image'' and ``mirror'' boundaries.}\label{fig:swisscheese_periodic}}
\end{figure}

\begin{figure}[!htpb]
 \centering
 \includegraphics{original_problem_rewrite}
 \caption{
   \csentence{Comparison of the RVE-problem formulations.}
   Original formulation (left) and Canonical formulation (right).\label{fig:format_comparison}}
\end{figure}

\begin{figure}[htb]
\centering
 \includegraphics[width=0.4\linewidth]{rve6.png}
 \includegraphics[width=0.4\linewidth]{rve6_inc.png}
\caption{\csentence{SVE-cube with sample realization of particle composite of size $6\times 6\times 6$.}\label{fig:rve_sample6}}
\end{figure}

\begin{figure}[htb]
\centering
 \includegraphics[width=0.4\linewidth]{rve9.png}
 \includegraphics[width=0.4\linewidth]{rve9_inc.png}
\caption{\csentence{SVE-cube with sample realization of particle composite of size $9\times 9\times 9$.}\label{fig:rve_sample9}}
\end{figure}

\begin{figure}[htb]
\centering
\includegraphics{meanG}
\includegraphics{varG}
\caption{\csentence{Statistical comparison of the influence of the boundary conditions (Dirichlet, Neumann) on the macroscale shear modulus, $\bar{G}$.}}
\label{fig:SVE_comp}
\end{figure}

\begin{figure}[htb]
\centering
 \includegraphics[width=0.5\linewidth]{rve6_def.png}
\caption{\csentence{A sample SVE subjected to macroscopic pure shear}. The prescribed macroscopic shear is $(\bar{\ts\epsilon}_\dev)_{ij} = 0.4[1 - \delta_{ij}]$.}
\label{fig:def_rve6}
\end{figure}

\begin{figure}[htbp!]
\centering
\includegraphics{CGmat}
\includegraphics{GGmat}
\caption{\csentence{Homogenized results from a single RVE with Dirichlet boundary condition.}
Dependence of effective properties $\bar{C}$ and $\bar{G}$ on the bulk compliance $C_\mathrm{mat}$ for fixed values of $G_\mathrm{part} = 5\,G_\mathrm{mat}$ and $C_\mathrm{part} = 0$.}
\label{fig:RVE_compressibility}
\end{figure}

%%%%%%%%%%%%%%%%%%%%%%%%%%%%%%%%%%%
%%                               %%
%% Tables                        %%
%%                               %%
%%%%%%%%%%%%%%%%%%%%%%%%%%%%%%%%%%%

%% Use of \listoftables is discouraged.
%%
% \section*{Tables}
% \begin{table}[h!]
% \caption{Sample table title. This is where the description of the table should go.}
%       \begin{tabular}{cccc}
%         \hline
%            & B1  &B2   & B3\\ \hline
%         A1 & 0.1 & 0.2 & 0.3\\
%         A2 & ... & ..  & .\\
%         A3 & ..  & .   & .\\ \hline
%       \end{tabular}
% \end{table}

%%%%%%%%%%%%%%%%%%%%%%%%%%%%%%%%%%%
%%                               %%
%% Additional Files              %%
%%                               %%
%%%%%%%%%%%%%%%%%%%%%%%%%%%%%%%%%%%

% \section*{Additional Files}
%   \subsection*{Additional file 1 --- Sample additional file title}
%     Additional file descriptions text (including details of how to
%     view the file, if it is in a non-standard format or the file extension).  This might
%     refer to a multi-page table or a figure.
% 
%   \subsection*{Additional file 2 --- Sample additional file title}
%     Additional file descriptions text.


\end{backmatter}
\end{document}
