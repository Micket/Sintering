\documentclass[a4paper,11pt]{article}
\usepackage[english]{babel}
\usepackage[utf8]{inputenc}
\usepackage{amsmath,amsfonts,amsthm,amssymb,graphicx,float,fancyhdr,caption,a4wide,listings}
\headheight=14.5pt
\pagestyle{fancy}
\fancyhead[r]{\today}
\fancyhead[l]{Mikael Öhman}
\newcommand{\ootimes}{\overline{\otimes}}
\newcommand{\uotimes}{\underline{\otimes}}
\newcommand{\tf}[1]{\text{\boldmath $\sf #1$}}
\newcommand{\ts}[1]{\text{\boldmath $#1$}}
\newcommand{\tv}[1]{\text{\boldmath $\sf #1$}}
\newcommand{\bi}[1]{\text{\boldmath $#1$}}
\newcommand{\pderiv}[2]{\frac{\partial #1}{\partial #2}}
\newcommand{\dderiv}[2]{\frac{\mathrm{d} #1}{\mathrm{d} #2}}
\providecommand{\abs}[1]{\lvert#1\rvert}
\newcommand{\defeq}{\stackrel{\rm def}{=}}
\newcommand{\dev}{{\rm dev}}
\newcommand{\vol}{{\rm vol}}
\newcommand{\sym}{{\rm sym}}
\newcommand{\tr}{{\rm tr}}
\newcommand{\rmd}{{\rm d}}
\newcommand{\rme}{{\rm e}}
\newcommand{\ATS}{\tf{E}_{\rm a}}
\DeclareMathOperator*{\Heaviside}{H}

\begin{document}
\section{Full form of epsilon}
Simplifying (7.37) and (7.39)
\begin{align}
 \ts\epsilon^\rme &= \ts\epsilon^{\rm e,tr} - \mu \ts\nu[\ts\sigma[\ts\epsilon^\rme]] \\
 \mu &= \frac{\Delta t}{t_\star} \eta[\Phi[\ts\sigma[\ts\epsilon^\rme]]]
\end{align}
with 
\begin{align}
 \Phi &= \sigma_\rme - \sigma_y\\
 \eta &= \left(\frac{\langle\Phi\rangle}{\sigma_c}\right)^{n}
\end{align}
and we obtain
\begin{align}
 \ts\epsilon^\rme &= \ts\epsilon^{\rm e,tr} - \mu \frac32 \frac{\ts\sigma_\dev}{\sigma_\rme}\\
\end{align}

We can now express the residual
\begin{align}
 \ts R_{\ts\epsilon}[\ts\epsilon^\rme,\mu] &= \ts\epsilon^\rme - \ts\epsilon^{\rm e,tr} + \mu \frac32 \frac{\ts\sigma_\dev}{\sigma_\rme}\\
 R_{\mu}[\ts\epsilon^\rme,\mu] &= \eta - \frac{t_\star}{\Delta t}\mu
\end{align}
as a function of the state 
\begin{align}
 \underline{X} = \begin{bmatrix}\ts\epsilon^\rme \\ \mu \end{bmatrix}
\end{align}
and the jacobian $\underline{J} = \pderiv{\underline{R}}{\underline{X}}$
\begin{align}
 \underline{J} = \begin{bmatrix} \tf J_{\ts\epsilon^\rme\ts\epsilon^\rme} & \ts J_{\ts\epsilon^\rme\mu} \\
                                 \ts J_{\mu\ts\epsilon^\rme}                 & J_{\mu\mu}\end{bmatrix}
\end{align}
where 
% Jee
\begin{align}
\nonumber \tf J_{\ts\epsilon^\rme\ts\epsilon^\rme} &= \pderiv{\ts R_{\ts\epsilon}}{\ts\epsilon^\rme} \\
\nonumber &= \tf I + \mu \frac32 \pderiv{}{\ts\epsilon^\rme}\left[\frac{\ts\sigma_\dev}{\sigma_\rme}\right]\\
          &= \tf I + \frac{3 G \mu}{\sigma_\rme}(\tf I_\dev -\frac32\frac{\ts\sigma_\dev\otimes\ts\sigma_\dev}{\sigma_\rme^2})
\end{align}
% Jue
\begin{align}
\nonumber \ts J_{\mu\ts\epsilon^\rme} &= \pderiv{\ts R_{\mu}}{\ts\epsilon^\rme} \\
\nonumber &= \frac{n}{\sigma_c} \left(\frac{\langle\Phi\rangle}{\sigma_c}\right)^{n-1} \pderiv{\langle\Phi\rangle}{\ts\epsilon^\rme}\\
\nonumber &= \Heaviside[\Phi]\frac{n}{\sigma_c} \left(\frac{\langle\Phi\rangle}{\sigma_c}\right)^{n-1} \pderiv{\sigma_\rme}{\ts\epsilon^{\rm e}}\\
 		  &= \Heaviside[\Phi]\frac{3 G n}{\sigma_c\sigma_\rme} \left(\frac{\langle\Phi\rangle}{\sigma_c}\right)^{n-1} \ts\sigma_\dev 		  
\end{align}
% Jeu
\begin{align}
\nonumber \ts J_{\ts\epsilon^\rme\mu} &= \pderiv{\ts R_{\ts\epsilon^\rme}}{\mu} \\
 &= -\frac32\frac{\ts\sigma_\dev}{\sigma_\rme}
\end{align}
%Juu
\begin{align}
\nonumber \ts J_{\mu\mu} &= \pderiv{R_\mu}{\mu}\\
 &= \frac{t_\star}{\Delta t}
\end{align}

\section{ATS tensor}

\begin{align}
 \ATS &= \ATS^\rme - \frac{1}{h_{\rm a}}\ATS^\rme\colon\ts\nu\otimes\ts\nu\colon\ATS^\rme\\
 h_{\rm a} &= \ts \nu \colon \ATS^\rme \colon \ts \nu + \left(\dderiv{\eta}{\Phi}\right)^{-1} \frac{t_\star}{\Delta t}\\
 \ATS^\rme &= ((\tf E_\rme)^{-1}+\mu\tf N)^{-1} = \\
 \tf N &= \frac{\partial^2\Phi}{\partial\ts\sigma\otimes\partial\ts\sigma}\\
       &= \pderiv{}{\ts\sigma} \left(\pderiv{\sigma_\rme}{\ts\sigma}\right)\\
       &= \pderiv{}{\ts\sigma} \left(\frac{\ts\sigma_\dev}{\sigma_\rme}\right)\\
       &= \pderiv{\ts\sigma_\dev}{\ts\sigma} \frac{1}{\sigma_\rme} + \ts\sigma_\dev\otimes\pderiv{\sigma_\rme^{-1}}{\ts\sigma}
\end{align}
and so on...

A simpler expression can be found for von mises effective stress
\begin{align}
 \tf E_a &= \tf E^\rme - 2 G b \tf Q - \frac{9G^2}{h_{\rm a} \sigma_\rme^2}\ts\sigma_\dev\otimes\ts\sigma_\dev\\
 		 &= \tf E^\rme - 2 G b \tf Q - \frac{4G^2}{h_{\rm a} (\epsilon^\rme_\rme)^2}\ts\epsilon^\rme_\dev \otimes \ts\epsilon^\rme_\dev\\
 b &= \frac{1}{\frac{\sigma_\rme}{3G\mu}+1}\\
   &= \frac{1}{\frac{\epsilon_\rme^\rme}{\mu}+1}\\
 \tf Q &= \tf I_\dev - \frac{3}{2\sigma_\rme^2}\ts\sigma_\dev\otimes\ts\sigma_\dev\\
 	   &= \tf I_\dev - \frac{2}{3(\epsilon_\rme^\rme)^2}\ts\epsilon_\dev\otimes\ts\epsilon_\dev\\
 h_a &= 3 G + \left(\dderiv{\eta}{\Phi}\right)^{-1}\frac{t_\star}{\Delta t}
\end{align}
For Norton material we obtain
\begin{align}
	h_{\rm a} &= 3 G + \frac{\sigma_c t_\star}{\Delta t}\left(\frac{\sigma_c}{\Phi}\right)^{n-1}
\end{align}
and for Bingham
\begin{align}
	h_{\rm a} &= 3 G + \frac{\sigma t_\star}{\Delta t}
\end{align}

\section{Reduced internal variables}

Another way is to use the effective elastic strain
\begin{align}
	\epsilon_\rme^\rme &= \sqrt{\frac23}\abs{\ts\epsilon^\rme_\dev}
\end{align}
where the stress can be calculated
\begin{align}
 \sigma_\rme = \sqrt{\frac32}\abs{\ts\sigma_\dev} = \sqrt{\frac32}\abs{2 G\ts\epsilon^\rme_\dev} = 3G\epsilon_\rme^\rme
\end{align}
which leads to
\begin{align}
\nonumber \ts\epsilon^\rme &= \ts\epsilon^{\rm e,tr}-\frac{\mu}{\epsilon_\rme^\rme}\ts\epsilon_\dev^\rme\\
\nonumber & \Longrightarrow\\
\nonumber \ts\epsilon^\rme_\dev &= \ts\epsilon^{\rm e,tr}_\dev-\frac{\mu}{\epsilon_\rme^\rme}\ts\epsilon_\dev^\rme\\
\nonumber & \Longrightarrow\\
 \left(1+\frac{\mu}{\epsilon_\rme^\rme}\right)\ts\epsilon^\rme_\dev &= \ts\epsilon^{\rm e,tr}_\dev
\end{align}
and taking the effective value we obtain
\begin{align}
	\epsilon_\rme^\rme + \mu &= \epsilon_\rme^{\rm e,tr}\\
	\mu &= \frac{\Delta t}{t_\star} \eta
\end{align}
\begin{align}
	R_\epsilon &= \epsilon_\rme^\rme -\epsilon_\rme^{\rm e,tr} +\mu\\
	R_\mu &= \eta - \frac{t_\star}{\Delta t}\mu\\
    \underline{J} &= \begin{bmatrix}1 & 1 \\ J_{\eta\epsilon} & -t_\star/\Delta t\end{bmatrix}\\
    J_{\eta\epsilon} &= \pderiv{\eta}{\epsilon_\rme^\rme} \label{eq:Jue}\\
    \underline{J}^{-1} &= \begin{bmatrix}...\end{bmatrix}
\end{align}
For Norton materials we obtain
\begin{align}
    \pderiv{\eta}{\epsilon_\rme^\rme} = \frac{3 G n}{\sigma_c} \left(\frac{\langle\Phi\rangle}{\sigma_c}\right)^{n-1}
\end{align}

And even further, we can also reduce $\mu$ completely;
\begin{align}
	R &= \epsilon_\rme^\rme -\epsilon^{\rm e,tr}_\rme + \frac{\Delta t}{t_\star}\eta\\
	J &= 1 + \frac{\Delta t}{t_\star}J_{\eta\epsilon}
\end{align}

The full elastic strain tensor can be calculated by scaling 
\begin{align}
	\ts\epsilon_\dev^\rme &= \frac{\epsilon_\rme^\rme}{\epsilon_\rme^\rme+\mu} \ts\epsilon^{\rm e,tr}_\dev
\end{align}

Under the condition $n=1$ we get 
\begin{align}
	0 &= \epsilon_\rme^\rme - \epsilon_\rme^{\rm e,tr} + \frac{\Delta t}{t_\star} \frac{\langle 3 G \epsilon_\rme^\rme - \sigma_y \rangle}{\sigma_c}\\
\nonumber &\Longrightarrow\\
	\epsilon_\rme^\rme &= \frac{\sigma_y\Delta t + \epsilon_\rme^{\rm e,tr}\sigma_c t_\star}{3 G \Delta t + \sigma_c t_\star}
\end{align}
and $\mu$, $\ts\epsilon_\dev^\rme$ can be calculated like before.

\subsection{ATS in the plastic region}
We have $R[\epsilon_\rme^\rme\{\ts\epsilon\},\ts\epsilon] = 0$, and linearizing this we obtain
\begin{align}
	%\rmd \epsilon_\rme^\rme &= \pderiv{\epsilon_\rme^\rme}{\ts\epsilon^\rme}\colon\rmd\ts\epsilon^\rme = 
	%	\frac23\frac{\ts\epsilon_\dev^\rme}{\epsilon_\rme^\rme}\colon\rmd\ts\epsilon^\rme \\
	\rmd R &= J \rmd\epsilon_\rme^\rme + \pderiv{R}{\ts\epsilon}\bigg|_{\epsilon_\rme^\rme}\colon\rmd\ts\epsilon = 0\\	
	\pderiv{R}{\ts\epsilon}\bigg|_{\epsilon_\rme^\rme} &= -\pderiv{\epsilon_\rme^{\rme,\rm tr}}{\ts\epsilon} = 
		-\pderiv{\epsilon_\rme^{\rme,\rm tr}}{\ts\epsilon_\dev^{\rme,\rm tr}}\colon\pderiv{\ts\epsilon_\dev^{\rme,\rm tr}}{\ts\epsilon} = 
		-\frac23\frac{\ts\epsilon_\dev^{\rme,\rm tr}}{\epsilon_e^{\rme,\rm tr}}
\end{align}
which gives
\begin{align}
	\rmd\epsilon_\rme^\rme &= 
		J^{-1}\frac23\frac{\ts\epsilon_\dev^{\rme,\rm tr}}{\epsilon_e^{\rme,\rm tr}}\colon\rmd\ts\epsilon
\end{align}

and we can obtain the relation
\begin{align}
	\rmd\ts\epsilon^\rme &= \rmd\ts\epsilon_\dev^\rme + \frac13\rmd\ts\epsilon_\vol \\
	\rmd\ts\epsilon_\vol &= (\ts I\otimes\ts I)\colon\rmd\ts\epsilon\\
	\rmd\ts\epsilon_\dev^\rme &= \pderiv{\ts\epsilon_\dev^\rme}{\epsilon_\rme^\rme}\bigg|_{\ts\epsilon}\rmd\epsilon_\rme^\rme 
		+ \pderiv{\ts\epsilon_\dev^\rme}{\ts\epsilon}\bigg|_{\epsilon_\rme^\rme}\colon\rmd\ts\epsilon \\
		&= \left(\pderiv{\ts\epsilon_\dev^\rme}{\epsilon_\rme^\rme}\bigg|_{\ts\epsilon}\otimes J^{-1}\frac23\frac{\ts\epsilon_\dev^{\rme,\rm tr}}{\epsilon_e^{\rme,\rm tr}}
		+ \pderiv{\ts\epsilon_\dev^\rme}{\ts\epsilon}\bigg|_{\epsilon_\rme^\rme}\right)\colon\rmd\ts\epsilon\\
	\pderiv{\ts\epsilon_\dev^\rme}{\epsilon_\rme^\rme}\bigg|_{\ts\epsilon} &= \frac{\mu-\mu'\epsilon_\rme^\rme}{(\epsilon_\rme^\rme+\mu)^2}\ts\epsilon_\dev^{\rme,\rm tr}\\
	\pderiv{\ts\epsilon_\dev^\rme}{\ts\epsilon}\bigg|_{\epsilon_\rme^\rme} &= \frac{\epsilon_\rme^\rme}{\epsilon_\rme^\rme + \mu} \tf I_\dev
\end{align}
with $\mu' = \frac{\Delta t}{t_\star}\pderiv{\eta}{\epsilon_\rme^\rme}$ which was shown earlier. 

Finally the ATS tensor as
\begin{align}
	\pderiv{\ts\epsilon^\rme}{\ts\epsilon} &= J^{-1} \frac23\frac{\mu-\mu'\epsilon_\rme^\rme}{\epsilon_\rme^{\rme,\rm tr}(\epsilon_\rme^\rme+\mu)^2}
		\ts\epsilon_\dev^{\rme,\rm tr}\otimes\ts\epsilon_\dev^{\rme,\rm tr} + 
		\frac{\epsilon_\rme^\rme}{\epsilon_\rme^\rme+\mu} \tf I_\dev + \frac13\ts I\otimes\ts I\\
	\ATS &= \tf E^\rme\colon\pderiv{\ts\epsilon^\rme}{\ts\epsilon}
\end{align}
\end{document}
