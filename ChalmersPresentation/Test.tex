\documentclass[12pt]{beamer}
\usepackage[utf8]{inputenc}
\usepackage[english]{babel}
\usepackage{tikz}
%\usepackage{beamerthemesplit}
\usetheme{chalmers}
\usepackage{amsmath}

% bullet=circle
% fancy front page, with image.

%\usefonttheme{professionalfonts}
%\usefonttheme{structurebold}

\title{Stability of High Speed Train under Aerodynamic Excitations}
%\title{Short title}
\author{Some Author}
\institute{Chalmers University of Technology}
\begin{document}

\section{Title page} % Sections are shown at the bottom left. There is also links in many pdf-readers
\begin{frame}[plain]
 \titlepage
\end{frame}

\begin{frame}
 \frametitle{Acknowledgements}
 We would like to say our thanks to
\begin{itemize}
 \item Department of Applied Mechanics
 \item Supervisors Sini\v{s}a Krajnovi\'{c} and Viktor Berbyuk, and Albin Johnsson
 \item CFD support and Fire licenses from AVL\\ with a special thanks to Dr. Branislav Basara
\end{itemize}
\end{frame}

\section{Introduction}
\begin{frame}
 \frametitle{Purpose and goal}
 \begin{itemize}
  \item To investigate the stability and comfort of riding a high speed train.
  \item For the vibration dynamics to have reasonably good aerodynamic data.
  \item Usage of moving meshes for doing CFD.
  \item Two scenarios; meeting trains and train coming out of tunnel with side wind.
  \item Low order mathematical model. Modular framework with functional components.
  \item Look at the sensitivity to speed and look at the effects of active damping, especially in the coupling.
 \end{itemize}
\end{frame}

\end{document}
