\documentclass{Eccomas}
\usepackage{graphicx}
\usepackage{amsmath}
\usepackage{amsfonts}
\usepackage{amssymb}
\usepackage{microtype}

\title{COMPUTATIONAL HOMOGENIZATION OF INCOMPRESSIBLE MICROSTRUCTURES}

\author{Mikael \"Ohman$^{*1}$, Kenneth Runesson$^{1}$ and Fredrik Larsson$^{1}$}

\heading{M. \"Ohman, K. Runesson and F. Larsson}

\address{
$^{1}$ Chalmers University of Technology, Department of Applied Mechanics, SE41296 G\"oteborg, mikael.ohman@chalmers.se, kenneth.runesson@chalmers.se, fredrik.larsson@chalmers.se\\
}

\keywords{FEM, Multiscale modeling, Incompressibility}

\begin{document}

One unsolved problem in computational homogenization occur when dealing with macroscopic incompressibility.
Such a situation is encountered for a composite whose micro-constituents are intrinsically incompressible, e.g.\ incompressible inclusions in an incompressible matrix, which infers macroscale incompressibility as well.
Alternatively, an initially compressible macroscale response may become incompressible as the result of the deformation process.
An important example is the evolving porous microstructure of a powder metallurgy product during the process of sintering, whereby the effective response is compressible until the porosity vanishes inferring incompressible macroscale response.
This process is thus characterized by a transition from the compressible to incompressible regimes that should be handled within the same variational framework, c.f. \cite{Ohman2013}.
In the case of an incompressible microstructure, the traditional homogenization schemes are not applicable.

In this submission we show how to derive the system of macro- and micro-scale equations for dealing with macroscopic incompressibility.
The macroscale problem is derived as a mixed formulation (displacement and pressure) system where the deviatoric stress and the volumetric strain are obtained through homogenization of a Representative Volume Element (RVE). The microscale problem is derived as a mixed formulation (displacement and pressure) problem combined with Lagrange multipliers associated with the choice of boundary conditions. We show the essential boundary conditions on the RVE; Periodicity, Dirichlet, and Neumann.

Examples of the new homogenization scheme are shown for incompressible elasticity and emphasis is put on the bounds given by the Dirichlet and Neumann boundary conditions.

% 
% Authors are invited to submit electronically their abstracts, through the Congress web site, before \textbf{November 29, 2013}. Abstracts should outline the main features, results and conclusions as well as their general significance, and contain relevant references.
% 
% The Abstract can be submitted directly in its final form. Authors will have the possibility of replacing the file by an updated version after the acceptance notification, but this will not be a requirement.
% 
% The Abstract should be written following the format of the Latex and Word macros for submission that can be found in the Registration and Abstract Submission area on: \emph{http://www.wccm-eccm-ecfd2014.org/frontal/Registration.asp}. The file must be converted to Portable Document Format (PDF) before submission through the Congress web site. Other formats are not accepted by the system.
% 
% The Abstract must be written in English within a printing box of 16cm x 21cm, centered in the page. The Abstract including figures, tables and references must not exceed 2 pages. Maximum file size is 4 MB.
% 
% The Abstract must contain the full name and full address of author/s. In the case of joint authorships, the name of the author who will actually present the paper at the Congress should be indicated with an asterisk. Papers can only be accepted on the understanding that they will be presented at the Congress.
% 
% Preliminary acceptance of the contribution will be communicated to the corresponding author by \textbf{January 31, 2014}.
% 
% The final Abstract in format for publication will be required by February 28th, 2014.
% 
% Please note that {\bf final acceptance} of papers for presentation is conditional to receiving the final Abstract and the payment of the presenting author's congress registration fee before {\bf February 28th, 2014}. Only one presentation per delegate is allowed.
% 
% {\bf Important:} Before submitting the contribution you should be registered. Please fill in the $\underline{\rm registration~form}$ choosing your login and password. To modify the information or add/modify the file of the Abstract you have to $\underline{\rm login}$. Registration does not oblige the author to pay the fees before having received the acceptance letter.
% 
% For any question, please contact WCCM-ECCM-ECFD 2014 Congress Secretariat.
% 
% E-mail: \emph{wccm-eccm-ecfd2014@cimne.upc.edu}


%
%   Finally the references done as a simple list,
%   rather than by a redefined bibliography environment...
%
%\vspace{0.5cm}

\begin{thebibliography}{99}
\setlength{\parskip}{0pt}

%\bibitem{OnCe93} E. O\~{n}ate and M. Cervera. Derivation of thin plate bending elements with one degree of freedom per node. {\em Engng. Comput.}, Vol. {\bf 10}, 543--561, 1993.

\bibitem{Ohman2013} M. \"Ohman, K. Runesson and F. Larsson. Computational Homogenization of Liquid-Phase Sintering with Seamless Transition from Macroscopic Compressibility to Incompressibility.
{\em Computer Methods in Applied Mechanics and Engineering}, Vol. {\bf 266}, 219--288, 2013.

%\bibitem{Zienkiewicz} O.C. Zienkiewicz,  R.L. Taylor. {\it The Finite Element Method\/}. Sixth Edition, Elsevier, 2005.


\end{thebibliography}

\end{document}


