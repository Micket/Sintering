\documentclass[11pt]{beamer} % mathserif for normal math fonts.
\usefonttheme[onlymath]{serif}
\usepackage[utf8]{inputenc}
\usepackage[swedish,english]{babel}
\usepackage{microtype}
\usepackage{calc}
\usepackage{amsmath,mathtools,dsfont}
\usepackage{contmech}
%\usepackage{siunitx}
%\usepackage{movie15}
\usepackage{multimedia}
\usepackage{grffile}
\usepackage{tikz}
\usepackage{pgfplots}
%\usepackage{subfig}
\usepackage{wasysym}

%\usepgfplotslibrary{patchplots}
%\usepgfplotslibrary{groupplots}
%\pgfplotsset{compat=1.3}

\newcommand{\roughcite}[1]{\textsc{#1}}
\renewcommand{\alert}[1]{\textbf{#1}}

\setbeamersize{text margin left=.3cm,text margin right=.3cm}

\usetheme[titleflower=false]{chalmers}
\title{Topics for discussion for future OOFEM development}
\author[Mikael \"Ohman]{Mikael \"Ohman}
\institute{Department of Applied Mechanics\\ Chalmers University of Technology}
\titlepageextra{Prague}% session: Multiple-scale physics and computation
\date{2013-04-11}
%\footer{\insertshortauthor\ 2\textsuperscript{nd} ICMM}
\titlepagelogofile{Avancez_black}

% Bibliography
%\bibliography{references_extended}

% Speeds up compilation.
% \usetikzlibrary{external}
% \tikzexternalize

\begin{document}

\section{Title page}
\begin{frame}[plain]
 \titlepage
\end{frame}

\section{Maintenance}
\begin{frame}
 \frametitle{OOFEM for new developers}
\begin{itemize}
 \item[\smiley] DVCS + Hierarchy of trust = Easier to get started
 \item[\smiley] CMake seems to be working quite well cross-platform
 \item[\smiley] Fairly decent documentation
 \item[\smiley] High abstraction layer
 \item[\smiley] Division of Material and Computational Mechanics at Chalmers is starting to use OOFEM increasingly (new PhD students incoming)
 \item[?] License
\end{itemize}
\begin{itemize}
 \item[\frownie] Need to de-mystify code
 \item[\frownie] Need to reduce the amount of code
 \item[\frownie] Not modular enough
 \item[\frownie] Better test coverage
 \item[\frownie] No info on who works on what until code is published
 \item[\frownie] Require author names on new code. Avoid commit-and-run
\end{itemize}
\end{frame}

\begin{frame}
 \frametitle{230 000 lines of code}
\begin{itemize}
 \item Mostly just base structures that matter
 \item Continuous evolving core design.
  \begin{itemize}
    \item Impossible to get everything right on the first try
    \item Old code needs some overhaul
    \item Rethink strategy if it comes down to cluttering base structure with specialized code
    \item Discuss with other developers for ideas
  \end{itemize}
 \item Try to play nice with existing code
 \item Keep it modular - Overwhelming size less of an issue
 \item If tests are unfeasible, then at least include an easy-to-run demo.
 \item Automatic tools are vital
  \begin{itemize}
   \item Test suite (+memcheck)
   \item Compiler warnings (MSVC is awful)
   \item CppCheck
   \item Code coverage
  \end{itemize}
 \item Minimize conditional compilation 
\end{itemize}
\end{frame}

\section{Selected issues}
% \begin{frame}
%  \frametitle{Parallel communication}
% \begin{itemize}
%  \item[\frownie] Too many classes!
%   \begin{itemize}
%    \item 
%   \end{itemize}
%  \itemize save/restoreContext and 
% \end{itemize}
% \end{frame}

\begin{frame}
 \frametitle{Too many engineering models}
\begin{itemize}
 \item[!] Always solve as nonlinear (linear converges in 0 or 1 iterations)
 \item Replace in SM-module;
  \begin{itemize}
   \item[-] DEIDynamic, DIIDynamic, IncrementalLinearStatic, LinearStatic, NonlinearStatic, NLDEIDynamic, NonLinearDynamic
   \item[+] Static, Dynamic (possibly separate ExplicitDynamic)
  \end{itemize}
 \item Replace in TM-module
  \begin{itemize}
   \item[-] Stationary, NonStationary, NLTransient
   \item[+] Stationary, Transient (possible separate ExplicitTransient)
  \end{itemize}
\end{itemize}
\end{frame}


\begin{frame}
 \frametitle{Questionable design patterns}
\begin{itemize}
 \item[\frownie] Common function + Enum + Switch-case
  \begin{itemize}
   \item Often completely unnecessary. Just write the direct function instead.
   \item To many obscure enum values
   \item Enum-blends; e.g. CharType mixed vectors and matrices.
   \item Design copy-pasted into new code unnecessarily
  \end{itemize}
\end{itemize}
\end{frame}

\section{Making life easier for pre-processors}

\begin{frame}
 \frametitle{Sets and boundary conditions}
\begin{itemize}
 \item Introduce sets, e.g. PointSet, BulkSet, BoundarySet, EdgeSet ...
 \item Sets contain lists of elements/nodes/edges
 \item Boundary condition applies to 1 (or more) set(s).
 \item Elements/nodes do \textbf{not} keep a list of BCs
 \item Essentially the way the ActiveBCs currently do it
 \item Assembly Neumann b.cs: Loop over BCs, loop over sets, loop over active elements only.
 \item Dirichlet b.c.s require inverse mapping for the dofs.
\end{itemize}
\begin{itemize}
 \item[\smiley] Much easier for adaptivity/remeshing
 \item[\smiley] No need for dofidmask in nodes, can be autodetected.
 \item[\smiley] Easier for pre-processors
 \item[\smiley] No need for the domain type record
 \item[\smiley] Only affected elements are requested for load assembly.
 \item[\smiley] Computations inside BC or inside Element can be optional
\end{itemize}
\end{frame}

\begin{frame}
 \frametitle{Made up example}
 \texttt{\footnotesize
  testset.out\\
  Test sets\\
  LinearStatic nsteps 1\\
  OutputManager tstep\_all dofman\_all element\_all\\
  ndofman 3 nelem 2 ncrosssect 1 nmat 1 nbc 2 nic 0 nltf 1 nsets 2\\
  Node 1 coords 2 0. 0.\\
  Node 2 coords 2 2. 0.\\
  Node 3 coords 2 1. 1.\\
  Truss2d 1 nodes 2 1 2 mat 1 crossSect 1\\
  Truss2d 2 nodes 2 2 3 mat 1 crossSect 1\\
  SimpleCS 1 thick 0.1 width 10.0\\
  IsoLE 1  tAlpha 0.000012  d 1.0  E 1.0  n 0.2 \\
  NodeSet 1 nodes 2 1 2\\
  NodeSet 2 nodes 1 3\\
  BoundaryCondition 1 TimeFunction 1 Set 1 d 2 0. 0. Dofs 2 D\_u D\_v\\
  Force 2 TimeFunction 1 Set 2 f 2 -1. 0.\\
  ConstantFunction 1 f(t) 1.0\\
}
\end{frame}

\section{The future}
\begin{frame}
 \frametitle{Post-processing}
 \begin{itemize}
  \item Replace Elixir with VTK.
   \begin{itemize}
    \item[\smiley] Can do much more than what UnstructuredGrid allows (IGA, XFEM, deformed gradient/beam elements).
    \item[\smiley] VTK already in use when writing binary VTU-files.
    \item[\smiley] Can use context files as output files.
    \item[\smiley] Plot solution while iterating
    \item[\frownie] Context files will almost always break when you have a slightly newer version of OOFEM
    \item[\frownie] A lot of work (but it can be added gradually)
   \end{itemize}
 \end{itemize}
\end{frame}

\end{document}
